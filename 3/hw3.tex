% Template by Jennifer Pan, August 2011

\documentclass[10pt,letter]{article}
	% basic article document class
	% use percent signs to make comments to yourself -- they will not show up.

\usepackage{amsmath}
\usepackage{amssymb}
	% packages that allow mathematical formatting

\usepackage{graphicx}
	% package that allows you to include graphics

\usepackage{setspace}
	% package that allows you to change spacing

\onehalfspacing
	% text become 1.5 spaced

\usepackage{fullpage}
	% package that specifies normal margins
	

\begin{document}
	% line of code telling latex that your document is beginning


\title{Problem Set 3}

\author{Tom O'Connell}

\date{August 20, 2016}
	% Note: when you omit this command, the current dateis automatically included
 
\maketitle 
	% tells latex to follow your header (e.g., title, author) commands.


\section*{1 Diffuse-to-Bounds Decision Making}

For each class that you are looking for in the world, you have a model $p_c(x_1,...,x_t | \theta_c)$ 
such that you can assign probabilities to sequences of observations, given the specific parameters 
($\theta$) of that model for class $c$.  Consider the case where there are two competing 
models with parameters $\theta_0$ and $\theta_1$.  Assuming independent measurements, we have 
to decide which of the following two forms is generating the measurements:\\

\[ p_0(x_1,...,x_t | \theta_0) = p(\theta_0)p(x_1 | \theta_0)...p(x_t | \theta_0)  \]
\[ p_1(x_1,...,x_t | \theta_1) = p(\theta_1)p(x_1 | \theta_1)...p(x_t | \theta_1)  \]

One approach to this decision would be be thresholding the likelihood ratio, in line with the 
Neyman-Pearson theorem. 

\[ \text{LR} = \frac{p_1(\theta_1 | x_1,...,x_t)}{p_0(\theta_0 | x_1,...,x_t)} \]

By Baye's rule, because the marginals cancel:

\[ \text{LR} = \frac{p(\theta_1)p_1(x_1,...,x_t | \theta_1)}
{p(\theta_0)p_0(x_1,...,x_t | \theta_0)} \]

However, we also want to be able to make this decision as soon as possible, again given the 
probabilities we will allow for false positives and false negatives 
($\alpha$ and $\beta$, respectively).  So we will check the cumulative likelihood ratio, $r$,
at each measurement, and stop taking measurements when we have enough information to decide 
beyond our preset thresholds.

\[ r = \frac{p(\theta_1)}{p(\theta_0)}\prod_{i=1}^{t}\frac{p(x_i | \theta_1)}{p(x_i | \theta_0)}\]

Now we just need to pick thresholds on this ratio.  For predetermined values of $\alpha$ and 
$\beta$, Wald proved that $T_1 \leq \frac{1 - \beta}{\alpha}$ and $T_2 \geq \frac{\beta}
{1 - \alpha}$ are tight bounds on the upper and lower thresholds we should use. Combined with 
our decision rule:

\[ \begin{cases}
   \text{Decide class 1} & r \geq T_1 \\
   \text{Decide class 0} & r \leq T_2 \\
   \text{Otherwise, continue}
   \end{cases}
\]

We now have a completely specified decision rule.  To update the likelihood ratio more easily, 
we will store the $\log$ likelihood ratio instead, and use the logs of our thresholds as well.  
Now each update is a sum instead of a product, and instead of $T_1$ and $T_2$, we will use their
logs.\\

A pair of neurons could implement this two threshold decision rule. Each would have to receive input 
from at least one neuron whose firing rate varies depending on which ground truth class is out in 
the world.  One would receive excitatory input from the tuned neuron, and it will integrate spikes 
it receives, firing (deciding on its class) if it receives enough input in a short period of time.  
Another neuron would receive inhibitory input from the tuned neuron, but otherwise drift closer to firing on its own.  If it receives a low enough level of inhibition and fires, it is deciding for its 
own class.

\section*{2 One-Neuron Decisions}

\paragraph{1)} Compute the false reject error rate when the lower decision threshold is

Need more information to compute these, such as $\alpha$, its complement,  or the upper decision
threshold. \\

The lower threshold is:

\[ B \leq \log(p(x=-1 | y=1) - \log(p(x=-1 | y = -1)) \]

This first quantity is the false negative rate (reject rate) and the second is unclear from this information.

\paragraph{2)} Compute how long it takes to decide GREEN with k action potentials when the decision
thresholds are

\subparagraph{a)} $\pm 1.2$\\

While there are no action potentials, an observer of the neuron would drift closer to the lower 
decision threshold at a rate of $(\ell_0 - \ell_1)\log_{10}(e) = (1 $Hz$ - 10 $Hz$)  0.434 = -3.906$
units per second. With a GREEN threshold at $-1.2$ units, it will only take $0.308$ seconds to 
decide GREEN without any action potentials.\\

Now with $k$ action potentials, each bringing us $\log_{10}(\frac{\ell_1}{\ell_0}) = \log_{10}(
\frac{10 \text{Hz}}{1 \text{Hz}}) = 1 $ units away from the lower decision threshold, 
it will take $0.308 + 0.256 k$ seconds 
for an observer of the cell to decide the (beyond the set threshold) that there is a GREEN light.\\

\subparagraph{b)} $\pm 1.8$\\

With the lower threshold of $-1.8$, it will take $\frac{1.8}{3.9} - \frac{k}{3.9} = 
0.462 + 0.256 k$ seconds 
for an observer of the cell to decide (beyond threshold) that there is a GREEN light.\\

\paragraph{3)} Compute how long it takes to decide RED (both upper and lower bounds) with decision
thresholds at $\pm 1.8$ using

%
%For the bounds, I will assume the tuned neuron is firing at a mean of 1 or 10 Hz, but obeying Poisson
 %statistics for spike generation. Given that there is RED and the mean firing rate should be 10 Hz:\\

%Some bounds, provided by the higher and lower number of spikes fulfilling the below relation:
% \[ p(\text{number of spikes in first second}) \approx 0.05 \]

%Lower bound, which will have more spikes, has 14 spikes in the first second. 
\subparagraph{a)} Assuming 2 action potentials happened, the lower bound for decision time is 
$\epsilon$ seconds, because the each spike will increment the cumulative log likelihood ratio by 
one log unit, since the firing rates differ by a factor of 10 depending on the true state of the world.\\

An upper bound is that no decision is ever reached in finite time. As long as the spikes happen far 
enough apart that the drift subtracts enough from the log likelihood ratio so that each of the three 
spikes fails to reach the upper decision threshold, then there is finite probability (in finite time) that the Poisson spiking statistics will never let the neuron fire enough to reach threshold.

\subparagraph{b)} Same as above. If they coincide, the decision could be reached almost immediately.\\

If you are really unlucky, it could take arbitrarily long to reach the decision.

\paragraph{4)} Use the SPRT code on Moodle to simulate the neuron’s decision. Confirm your calculated
decision times from 2.3.a and 2.3.b. Provide the generated figures in your
response.\\

If we ran the simulation enough times, you would start to see more of a tail to the distribution of 
response times, but you can already see some high latency decisions in the $C = 1$ case.\\

It is also apparent from the response time distributions that some times the spikes will coincide 
enough early on to get a very fast response time.  In reality, you would be limited by the rarity 
of such rapid spiking, and by the refractory period of the neuron.\\

See graphs attached at the bottom.

\section*{3 Two-Neuron Decisions}

\paragraph{1)} Two identically tuned neurons, uncorrelated, tuned as above.\\

With two uncorrelated neurons, the drifts would just happen twice as fast because the thresholds
 would be the same, but you sum the log likelihood ratios from each on each time step.  Individual
 spikes still only contribute the same amount.

\subparagraph{2.2.a)} $0.616 + 0.256 k$ seconds 

\subparagraph{2.2.b)} $0.914 + 0.256 k$ seconds 

\subparagraph{2.3)} The answer for 2.3 does not change. It is now more unlikely that you will have 
rare events prevent you from reaching the RED decision, but it could still happen for less than 
infinite time.  Additionally since the thresholds are the same and each action potential contributes 
equally still, very fast decisions can also be reached.

\paragraph{2)} One neuron tuned as above, and one tuned similarly for green rather than 
    red. Uncorrelated. \\

Now the drifts will cancel out since the neurons are tuned in opposite directions, assuming both 
cells are silent.  and it will only 
be spiking from one or both of the neurons that can bring the system closer to threshold.

\subparagraph{2.2.a)} For $k > 1$, on the part of the GREEN tuned cell, it is possible to reach the 
decision threshold for GREEN very quickly, depending on the relative timing of the spikes.

\subparagraph{2.2.b)} Same as above.

\subparagraph{2.3)} Same, again. This case will generally reach RED decisions with an expected time 
slightly longer than above, but will be able to reach GREEN decisions sooner, because it does not 
have to rely wholly on drift.

\end{document}
	% line of code telling latex that your document is ending. If you leave this out, you'll get an error
