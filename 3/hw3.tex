% Template by Jennifer Pan, August 2011

\documentclass[10pt,letter]{article}
	% basic article document class
	% use percent signs to make comments to yourself -- they will not show up.

\usepackage{amsmath}
\usepackage{amssymb}
	% packages that allow mathematical formatting

\usepackage{graphicx}
	% package that allows you to include graphics

\usepackage{setspace}
	% package that allows you to change spacing

\onehalfspacing
	% text become 1.5 spaced

\usepackage{fullpage}
	% package that specifies normal margins
	

\begin{document}
	% line of code telling latex that your document is beginning


\title{Problem Set 3}

\author{Tom O'Connell}

\date{August 20, 2016}
	% Note: when you omit this command, the current dateis automatically included
 
\maketitle 
	% tells latex to follow your header (e.g., title, author) commands.


\section*{1 Diffuse-to-Bounds Decision Making}

For each class that you are looking for in the world, you have a model $p_c(x_1,...,x_t | \theta_c)$ 
such that you can assign probabilities to sequences of observations, given the specific parameters 
($\theta$) of that model for class $c$.




\section*{2 One-Neuron Decisions}

\paragraph{1)} Compute the false reject error rate when the lower decision threshold is

\subparagraph{a)} $-1$

\subparagraph{b)} $-1.2$

\subparagraph{c)} $-1.8$

\paragraph{2)} Compute how long it takes to decide GREEN with k action potentials when the decision
thresholds are

\subparagraph{a)} $\pm 1.2$\\

While there are no action potentials, an observer of the neuron would drift closer to the lower 
decision threshold at a rate of $(\ell_0 - \ell_1)\log_{10}(e) = (1 $Hz$ - 10 $Hz$)  0.434 = -3.906$
units per second. With a GREEN threshold at $-1.2$ units, it will only take $0.308$ seconds to 
decide GREEN without any action potentials.\\

Now with $k$ action potentials, each bringing us $\log_{10}(\frac{\ell_1}{\ell_0}) = \log_{10}(
\frac{10 \text{Hz}}{1 \text{Hz}}) = 1 $ units away from the lower decision threshold, 
it will take $0.308 + 0.256 k$ seconds 
for an observer of the cell to decide the (beyond the set threshold) that there is a GREEN light.\\

\subparagraph{b)} $\pm 1.8$\\

With the lower threshold of $-1.8$, it will take $\frac{1.8}{3.9} - \frac{k}{3.9} = 
0.462 + 0.256 k$ seconds 
for an observer of the cell to decide (beyond threshold) that there is a GREEN light.\\

\paragraph{3)} Compute how long it takes to decide RED (both upper and lower bounds) with decision
thresholds at $\pm 1.8$ using

\subparagraph{a)} 

\subparagraph{b)} 

\paragraph{4)} Use the SPRT code on Moodle to simulate the neuron’s decision. Confirm your calculated
decision times from 2.3.a and 2.3.b. Provide the generated figures in your
response.

\section*{3 Two-Neuron Decisions}

\paragraph{1)} Two identically tuned neurons, uncorrelated, tuned as above.

\subparagraph{2.2.a)} 

\subparagraph{2.2.b)} 

\subparagraph{2.3.a)} 

\subparagraph{2.3.b)} 

\paragraph{2)} One neuron tuned as above, and one tuned similarly for green rather than 
    red. Uncorrelated.

\subparagraph{2.2.a)} 

\subparagraph{2.2.b)} 

\subparagraph{2.3.a)} 

\subparagraph{2.3.b)} 



\end{document}
	% line of code telling latex that your document is ending. If you leave this out, you'll get an error
